\chapter{Conclusion and Discussion}

In this final chapter, we reflect on the journey undertaken throughout this thesis,
drawing together the threads of our research and exploring their broader significance.
From the initial exploration of radar imaging algorithms to the in-depth analysis of their performance,
each step has brought us closer to a deeper understanding of radar technology and its applications.

\section{Overview}
\autoref{ch:intro} of this thesis delineated two overarching objectives
aimed at advancing radar imaging technology. Firstly, it sought to identify 
and evaluate a novel imaging algorithm capable of surpassing the limitations inherent
in the current FFT-based approach developed for the system under test. Specifically, 
the chapter aspired to develop an algorithm harnessing measured antenna gains to 
enhance imaging precision and fidelity significantly. 
Secondly, the introduction provided a comprehensive overview of radar technology's pivotal role,
particularly within the domain of mining applications. Through elucidating radar's pivotal role 
in streamlining bulk logistics and bolstering security measures,
the chapter underscored the multifaceted applications of radar technology.
% Additionally, the chapter recognized the inherent variability in radar research due to hardware discrepancies and outlined the thesis's broader implications for advancing FMCW MIMO radar imaging technology.
These advancements hold promise for transformative applications across remote sensing,
autonomous vehicles, and surveillance.
\\
In \autoref{ch:fundamentals}, foundational concepts essential for 
radar technology and imaging algorithms were explored. 
The chapter began with Antenna Fundamentals, introducing key parameters.
It then provided a detailed framework in the Signal Model section,
laying the groundwork for understanding signal processing in FMCW MIMO radar imaging 
and setting the stage for discussions on imaging algorithms, including the versatile Backprojection technique.
\\
The hardware evaluation in \autoref{ch:hardware} delved into the meticulous examination
of the iMCR sensor, assessing its stability and performance under various conditions.
Through systematic measurements and analysis, the chapter provided valuable insights 
into the reliability and functionality of the hardware components,
as well as the applicability of the proposed imaging algorithms.
\\
Embarking on a comprehensive exploration of radar imaging algorithms in \autoref{ch:imaging},
we evaluated their performance and efficacy in extracting meaningful information from radar data.
Leveraging the insights gained from stability analysis and antenna gain measurements
conducted in previous sections, a detailed examination of FFT-based imaging,
Backprojection, and a hybrid approach was conducted.
Through a systematic comparison of the resulting images and analysis of key metrics
such as the azimuth and elevation peak widths, a deeper understanding of the capabilities
of each algorithm was obtained.

\section{Key Findings}
This section encapsulates the pivotal discoveries extracted from both the hardware evaluation and imaging exploration.
In Chapter 3, the thesis rigorously examines the stability and antenna characteristics of the radar hardware,
uncovering fundamental insights crucial for radar imaging. These insights are complemented by the findings from Chapter 4,
which delve into the performance and efficacy of various radar imaging algorithms. 

The stability analysis of the radar system revealed consistent performance over time,
with gradual changes observed in system parameters such as temperature and runtime.
Frequent recalibrations were identified as essential for maintaining optimal system performance.

Antenna gain measurements provided valuable insights into the signal model's accuracy,
confirming the feasibility of extracting antenna characteristics even from imperfect measurements.

Putting the imaging algorithms to the test revealed that,
while the FFT-based approach offers the fastest runtime, Backprojection shows promising potential
for improved image fidelity, particularly in reducing distortion in the near-field
and enhancing signal-to-noise ratio (SNR). Furthermore, the hybrid approach demonstrates
comparable performance to the FFT, suggesting real-time capability with further optimization.
These results underscore the importance of selecting the appropriate imaging algorithm based
on specific application requirements, balancing considerations such as runtime efficiency
and image quality.

\section{Outlook and Discussion}

Concluding our investigation into real-time capable calibration and image reconstruction techniques for FMCW MIMO radars,
it's essential to reflect on the significant insights garnered and explore avenues for future research.
In this section, we discuss the implications of our findings, potential areas for improvement,
and outline future directions in the field.
\\

Exploring the stability of our experimental setup reveals subtle yet impactful influences on radar performance.
Minor variations in temperature and humidity levels cause nuanced shifts within the system, 
gradually affecting measurement accuracy over time.
Minimizing these environmental dynamics is crucial for better understanding the systems stability.

Further investigation into self-calibration mechanisms within the radar frontend would also improve our understanding of system stability.
While designed to maintain system integrity, the inner workings of self-calibration remain somewhat opaque.
Clarifying its duration, mechanisms, and effectiveness is an essential next step.

Temperature regulation emerges as another critical consideration for the stability measurements.
Exploring controlled cooling measures, such as strategic cooler spray application, 
promises to highlight temperature-induced distortions in radar measurements.
By observing corresponding changes in phase stability, 
researchers can develop robust temperature compensation strategies, 
ensuring consistent radar performance across diverse environments.

\\
In antenna gain measurements, integrating an external reference antenna
holds promise for enhancing measurement accuracy and precision.
This approach enables absolute measurements, elevating the reliability of antenna characterization processes.

Meanwhile, exploring different radome configurations offers insights into fine-tuning antenna performance, 
vital for optimizing performance in dynamic environmental conditions.

\\
Shifting focus to imaging algorithms, bridging the conceptual gap between FFT-based imaging and 
backprojection techniques presents exciting opportunities.
Adapting windowing concepts from FFT-based imaging to backprojection could enhance image accuracy and fidelity.
Also, to enhance the DOA capabilities of the backprojection algorithm,
the array's diversity can be improved through appropriate windowing.

Implementing adaptive sampling techniques in image reconstruction offers promise for optimizing
computational resources and enhancing real-time imaging capabilities.
Regions of interest can be identified from low-resolution images,
making it possible to adaptively increase the resolution only where it matters.
\\

In summary, proposed improvements in stability analysis,
antenna gain measurements, and imaging algorithms represent significant strides in improving imaging for FMCW MIMO radar systems.
By embracing a culture of continuous improvement and fostering curiosity-driven research, 
researchers are poised to propel radar technology forward,
driving incremental advancements that collectively revolutionize radar applications across diverse domains.

