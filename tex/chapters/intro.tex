\chapter{Introduction}
\section{Theoretical Background}
\subsection{Signal Model}
\subsubsection*{Single Channel FMCW}
A single channel consists of a transmit antenna and a receive antenna.
The transmit antenna sends a so-called chirp of duration $T_{chirp}$,
which is a sinusoid with linearly increasing frequency.
The signal $x_{TX}(t)$ send by the transmit antenna is reflected by an ideal point scatterer at position $\vec r_S$
and then received at the receive antenna as $x_{RX}(t)$.
The propagation delay $\tau$ can be calculated using the speed of light $c_0$,
and the locations of the receive and transmit antennas $\vec r_{RX}$ and $\vec r_{TX}$:
\begin{align}
    \tau = \frac{\| \vec r_{TX} - \vec r_S \|+\| \vec r_{RX} - \vec r_S \|}{c_0}
\end{align}
Using a complex representation for the in-phase and quadrature components of the signal,
the transmit and receive signal can be formulated for $t \in [0, T_{chirp}]$:
\begin{align}
    x_{TX}(t) & = A_0 e^{j(\omega_0t + \frac{1}{2}\dot \omega t^2 + \phi_0)} \label{eqn:x_TX} \\
    x_{RX}(t) & = A(\vec r_S) x_{TX}(t-\tau)                                 \label{eqn:x_RX} \\
\end{align}
The received signal is then mixed with a copy of the transmitted signal (\ref{eqn:x_TX}) and a low-pass filter is applied.
The resulting signal $y(t)$ is called \textit{intermittent frequency} signal.
\begin{align}
    y(t) & = \text{LP} \left\{ x_{RX}(t) \cdot x_{TX}(t) \right\} \\
         & = \text{LP} \left\{
    A_0 e^{j(\omega_0t + \frac{1}{2}\dot \omega t^2) }
    \cdot A(\vec r_S) A_0 e^{j(\omega_0(t-\tau(\vec r)) + \frac{1}{2}\dot \omega (t-\tau(\vec r))^2) }
    \right\}                                                      \\
         & = A_0^2A(\vec r_S)
    e^{j(\frac{1}{2}\dot\omega\tau^2(\vec r_S)- \omega_0\tau)}
    \cdot  \text{LP} \left\{
    e^{j(2\omega_0 t + \frac{1}{2}\dot\omega t^2 - \dot\omega\tau t)}
    \right\}                                                      \\
         & \approx G(\vec r_S) e^{-j\dot\omega\tau t}
\end{align}
The fact that the IF-signal contains all the information
-- i.e. the IF signal's frequency directly corresponds to the target's distance --
explains the main advantage of this technology.
The carrier frequency can be orders of magnitude higher than the intermittent frequency,
which drastically reduces the requirements for the subsequent signal processing,
while retaining the improved resolution due to the smaller wavelenghts of the carrier frequency.

FIND QUOTE: "GHz resolution for MHz processing"

To locate a target in the cross-range dimensions,
a single-channel FMCW-radar can be used to scan in multiple directions,
by either rotating the antennas, redirecting their beam with rotating mirrors, or with beamforming antenna arrays.
In any case, this requires highly directive antennas and also increases size, weight and cost of a radar sensor.

\subsubsection*{MIMO FMCW}
Multiple-input multiple-output radar benefits from increased diversity and signal power.
If $M$ transmit antennas and $N$ receive antennas are employed, $K=M \cdot N$ channels are available.
To differentiate the signals from each other, a multiplexing technique has to be chosen.
Options include time division multiplex, frequency division multiplex and code division multiplex. \\

In TDM, multiple access is achieved by the transmit antennas all send one after another,
while all receive antennas receive simultaneously.
In FDM, simultaneous transmission is made possible by subdividing the bandwidth and assigning a different frequency range to each antenna.
That means that TDM allows for higher bandwidths for each transmission, while FDM allows higher transmission durations.

In CDM, both simultaneous transmission and use of the entire bandwidth is made possible by using a different waveform to each channel.
However, processing at the carrier frequency is required to differentiate the signals from another, as opposed to TDM and FMD,
where all processing can be done at the intermittent frequency range.

Depending on the application, a compromise has to be found between the advantages and drawbacks of each method.
There are also methods available that combine aspects of these three basic paradigms, such as OFDM and Hadamard-Coding. \\

Once the received signals are demultiplexed, the ideal receive signal for antenna pair $k \in \{0,1,...K-1\}$:

\begin{align}
    y_k(t) & = G_k(\vec r_S)e^{-j\dot\omega\tau_k(\vec r_S)t}
\end{align}
Note that both the gain and the propagation delay may differ from channel to channel.

In reality, the scene can consist of multiple and expansive scatterers,
that reflect the transmitted signals at different intensities.
which is summarized as a locational reflectivity $F_k(\vec r)$.
\footnote{
    The index $k$ is introduced here to take obstructed visibility into account:
    from the point of view of one channel, two scatterers may be visible simultaneously,
    while from the point of view of another, one might obstruct the other's visiblity.
}
Also, interference and electric noise may be present in each channel,
which we summarize as $n_k(t)$.
Thus, the overall IF-signal is:
\begin{align}
    y_k(t) & = \iiint F_k(\vec r)G_k(\vec r) e^{-j\dot\omega\tau_k(\vec r)t} \;d\vec r + n_k(t) \\
\end{align}
After sampling the signal at sampling intervals $T_s$ such the sampling frequency $f_s = \frac{1}{T_s}$
is sufficiently high: ${2f_s > \frac{1}{2\pi}(\omega_0 + \dot \omega T_{chirp})}$, and with $M$ samples such that $MT_s < T_{chirp}$,
the sampled IF-signal can be defined as:
\begin{align}
    y_k[m] = y_k(t=mT_s), \text{for}\;m \in \{0,1,..M-1\}
\end{align}
\subsection{Calibration}

\subsection{Image Reconstruction}
Image reconstruction is the inverse problem of estimating the locational reflectivity of the scene $F(\vec r)$
from the received signals $y_k[m]$. Multiple approaches are available; in the following, three will be presented.

\subsubsection*{Discrete Fourier Transform}
The discrete fourier transform can be implemented with high efficiency, and many CPUs even include silicone-based implementations.
In this approach, the DFT is applied over three dimensions of the input signal,
obtaining a discrete output signal in spherical coordinates whose amplitude is an estimate of the locational reflectivity.

For each input channel, the range of a target can be estimated by applying the DFT over time.
The resulting spectrum's peak corresponds to the target:
\begin{align}
    \mathcal{F}\{y_k(t)\}(\omega) = G_k(\vec r_S) \delta(\omega-\dot \omega \tau_k(\vec r_S))
\end{align}

The cross-range dimensions can also be calculated using the FFT, but some approximations have to be made.
Suppose $K_\theta \subset \{0,1,...,K-1\}$ is a set of channels
\section{Physical Setup}

