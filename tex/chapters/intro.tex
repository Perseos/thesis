\chapter{Introduction}
\label{ch:intro}
Radar technology has undergone significant advancements over the decades,
revolutionizing various fields including aerospace, defense, automotive, and healthcare.
Its fundamental principle of using electromagnetic waves to detect the presence,
direction, distance, and speed of objects has enabled myriad applications, ranging from weather monitoring to target tracking.

Within this expansive domain, Frequency Modulated Continuous Wave (FMCW)
Multiple Input Multiple Output (MIMO) Millimeter Wave (mmWave) radar systems have emerged
as a cutting-edge solution with unparalleled capabilities.
FMCW radar, characterized by its continuous transmission of frequency-modulated signals,
combined with MIMO architecture, which utilizes multiple antennas for both transmission and reception,
has enabled enhanced spatial resolution, improved target detection, and increased resilience to interference.
Operating in the mmWave spectrum, typically within the frequency range of 24 to 100 GHz,
these radars offer advantages such as high resolution, immunity to environmental conditions
like fog and precipitation, and the ability to accommodate large bandwidths for high data rates.

The fusion of FMCW, MIMO, and mmWave technologies represents a significant leap forward in radar sensing,
promising to address the evolving demands of modern applications ranging from autonomous vehicles to biomedical imaging.
This thesis endeavors to contribute to this dynamic field by proposing and evaluating
novel imaging algorithms optimized for FMCW MIMO mmWave radar systems, with a focus on
enhancing their performance, efficiency, and applicability across diverse scenarios and applications. \\

\section{Radar Applications in Mining}
Radar technology finds diverse applications across various industries, with one notable area being mining.
In the mining sector, radar systems play a crucial role in enhancing safety, efficiency, and productivity.

Radar-based collision avoidance systems are employed in mining operations to prevent accidents involving heavy machinery and personnel.
These systems utilize radar sensors to detect nearby objects and provide real-time alerts to operators,
facilitating safe maneuvering in challenging environments.

Moreover, radar technology is instrumental in various logistical aspects of mining operations.
It is utilized for stockpile management, enabling accurate measurement and monitoring of ore and waste material reserves,
optimizing inventory control and resource allocation.
Radar sensors are also integrated into silo and train loading processes, ensuring efficient and precise transfer of bulk materials onto transport vehicles.
Furthermore, radar-based systems facilitate ship unloading and berthing operations at port facilities, streamlining the handling of bulk commodities.
One significant advantage of radar systems over visual-based solutions in mining environments is their resilience to omnipresent dust and grime.
Unlike optical systems, radar sensors are not affected by adverse weather conditions or obstructed visibility,
ensuring reliable performance and continuous operation even in harsh mining environments.\\

\section{MIMO Radar Sensors}
In the realm of radar sensing, various approaches exist for achieving multidimensionality in imaging.
Traditional radar systems employ one-dimensional (1D) sensors, which provide information along a single axis, typically range.
These sensors are adept at measuring distances to targets but lack detailed spatial information.
To overcome this limitation, two-dimensional (2D) sensors have been developed,
capable of scanning in both range and azimuth dimensions, providing a more comprehensive view of the surrounding environment.
These sensors often achieve multidimensionality either mechanically,
by utilizing mechanisms such as rotating antennas or electronically with phased arrays to sweep the radar beam across the scene.
Additionally, three-dimensional (3D) radar sensors offer even greater spatial resolution by adding an elevation dimension to the azimuth and range measurements.
These sensors are commonly used in applications requiring detailed volumetric imaging.
The mechanical movement or rotation of the sensor enables it to scan the scene from multiple angles, capturing information in three dimensions.\\

The evolution of radar technology has led to the development of advanced imaging techniques
that leverage the principles of Multiple Input Multiple Output (MIMO) radar systems.
These systems, characterized by their use of multiple transmit and receive antennas,
offer significant improvements in imaging performance compared to traditional radar architectures.
One key advantage of MIMO radar is its ability to achieve multidimensional imaging without the need for
mechanically rotating antennas. By utilizing multiple antennas in a static configuration,
MIMO radar systems can capture spatial information along multiple dimensions simultaneously,
leading to enhanced imaging capabilities at reduced cost and complexity. Additionally,
MIMO radar systems offer the potential for higher frame rates compared to traditional radar systems.
With multiple antennas operating in parallel, MIMO radar can sample the scene more frequently,
enabling rapid data acquisition and real-time imaging of dynamic environments.
The next step in the advancement of MIMO imaging radars involves the development and optimization
of sophisticated signal processing algorithms tailored to exploit the full potential of these systems.
These algorithms aim to extract rich spatial information from the received signals,
enabling high-fidelity imaging of complex scenes with unprecedented detail and accuracy. \\

\section{Imaging Algorithms for 3D MIMO Radar Sensors}
Current 3D imaging algorithms employed for many FMCW MIMO radars rely on the Fast Fourier Transform (FFT)
technique for generating three-dimensional reconstructions of the observed scene.
This algorithm computes each dimension of the image by performing an FFT on a different dimension
of the Intermediate Frequency (IF) signal obtained from the radar's input data.

% Specifically, the range dimension corresponds to the time dimension of the input signal,
% while the azimuth and elevation dimensions are computed from horizontal and vertical channels, respectively.
One of the primary strengths of this approach lies in its fast implementation,
allowing for rapid processing of radar data. However, despite its efficiency,
the FFT-based algorithm suffers from several limitations.

These include near-field distortion issues,
restrictions in applicability to Uniform Linear Arrays (ULA),
the necessity for array calibration, and the inability to utilize measured antenna gains effectively.
Moreover, the algorithm is prone to ringing artifacts, which can degrade the quality of reconstructed images,
particularly in complex or cluttered environments. \\

In response to the limitations of the current FFT-based approach,
a novel back projection algorithm is proposed for 3D imaging in the FMCW MIMO radar system.
Unlike the FFT-based method, the proposed algorithm offers increased flexibility and versatility in image reconstruction.
One of its key advantages is the elimination of ringing artifacts, which commonly plague FFT-based reconstructions.
Additionally, the back projection algorithm is capable of addressing near-field distortion issues and is suitable
for both near- and far-field imaging scenarios. Moreover, it exhibits compatibility with non-ULA arrays,
enabling broader applicability across different radar configurations. A notable feature of the proposed algorithm
is its ability to incorporate measured antenna gains effectively, leading to more accurate and reliable imaging results.
By leveraging these advantages, the back projection algorithm promises to significantly enhance the imaging capabilities of the FMCW MIMO radar,
paving the way for improved performance and expanded applications in various domains. \\

\section{Thesis Outline}
In this thesis, we embark on a comprehensive exploration of advanced imaging techniques for FMCW MIMO radar systems,
aiming to address existing challenges and enhance imaging performance.

Chapter 1 provides a thorough examination of
the theoretical background underpinning radar imaging, including signal models and a detailed description of the image reconstruction methods.
We delve into the intricacies of both FFT-based and proposed back projection algorithms.
Furthermore, a detailed hardware description of the FMCW MIMO radar system under investigation is presented,
laying the foundation for subsequent experimental chapters.

Chapter 2 focuses on measurements and validation of hardware stability
through extensive static measurements. Through rigorous analysis, we assess the stability of the radar system and measure antenna gains,
crucial for accurate imaging.

Chapter 3 shifts the focus to imaging algorithms, detailing the implementation of the back projection algorithm using PyTorch.
Additionally, we discuss the interpolation of measured gains and evaluate the performance enhancements achieved through these refinements.

By presenting comprehensive results and analyses, this thesis aims to contribute to the advancement of FMCW MIMO radar imaging technology,
with implications for various applications in fields such as remote sensing, autonomous vehicles, and surveillance.

